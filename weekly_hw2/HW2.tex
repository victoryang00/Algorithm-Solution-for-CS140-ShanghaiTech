%=======================02-713 LaTeX template, following the 15-210 template==================
%
% You don't need to use LaTeX or this template, but you must turn your homework in as
% a typeset PDF somehow.
%
% How to use:
%    1. Update your information in section "A" below
%    2. Write your answers in section "B" below. Precede answers for all 
%       parts of a question with the command "\question{n}{desc}" where n is
%       the question number and "desc" is a short, one-line description of 
%       the problem. There is no need to restate the problem.
%    3. If a question has multiple parts, precede the answer to part x with the
%       command "\part{x}".
%    4. If a problem asks you to design an algorithm, use the commands
%       \algorithm, \correctness, \runtime to precede your discussion of the 
%       description of the algorithm, its correctness, and its running time, respectively.
%    5. You can include graphics by using the command \includegraphics{FILENAME}
%
\documentclass[11pt]{article}
\usepackage{amsmath,amssymb,amsthm}
\usepackage{listings}
\usepackage{graphicx}
\usepackage{url}
\usepackage{amsfonts}
\usepackage[english]{babel}
\usepackage{framed}
\usepackage{enumerate}
\usepackage[margin=1in]{geometry}
\usepackage{fancyhdr}
\newtheorem{Q}{Question}
\setlength{\parindent}{0pt}
\setlength{\parskip}{5pt plus 1pt}
\setlength{\headheight}{13.6pt}
\newcommand\question[2]{\vspace{.25in}\hrule\textbf{#1: #2}\vspace{.5em}\hrule\vspace{.10in}}
\renewcommand\part[1]{\vspace{.10in}\textbf{(#1)}}
\newcommand\algorithm{\vspace{.10in}\textbf{Algorithm: }}
\newcommand\correctness{\vspace{.10in}\textbf{Correctness: }}
\newcommand\runtime{\vspace{.10in}\textbf{Running time: }}
\pagestyle{fancyplain}
\lhead{\textbf{\NAME\ \ \ \ \ \ \ \ \ \ \ \ \ANDREWID}}
\chead{\textbf{HW\HWNUM}}
\rhead{Due date: 11:59 PM, September 29, 2019}
\begin{document}\raggedright
%Section A==============Change the values below to match your information==================
\newcommand\NAME{Your Name:}  % your name
\newcommand\ANDREWID{Your ID:}     % your andrew id
\newcommand\HWNUM{3}              % the homework number
%Section B==============Put your answers to the questions below here=======================

% no need to restate the problem --- the graders know which problem is which,
% but replacing "The First Problem" with a short phrase will help you remember
% which problem this is when you read over your homeworks to study.

\section{(5') Stack, Queue and Complexity Analysis}
Each question has one or more correct answer(s). Select all the correct answer(s). For each question, you get $0$ point if you select one or more wrong answers, but you get $0.5$ point if you select a non-empty subset of the correct answers.\\
\textit{Note that you should write you answers of section 1 in the table below.}
\begin{table}[htbp]
	\begin{tabular}{|p{2cm}|p{2cm}|p{2cm}|p{2cm}|p{2cm}|}
		\hline 
		Question 1 & Question 2 & Question 3 & Question 4 & Question 5  \\ 
		\hline 
		&  &  &  & \\ 
		\hline 
	\end{tabular} 
\end{table}
\begin{Q}
	In the lectures of Week 2, suppose we implement a circular queue by using an array with the index range from $1$ to $n$, then what the size of this queue would be? We assume that the queue is non-empty.
	\begin{enumerate}[(A)]
		\item $rear-front+1$
		\item $(rear-front+1)\%n$
		\item $(rear-front+n)\%n$
		\item $(rear-front+n)\%n+1$
	\end{enumerate}
\end{Q}


\begin{Q}
	Which of the following is known to be correct? 
	\begin{enumerate}[(A)]
		\item Stack is a linear data structure and the operations on stacks are more restricted, the same is true for queue.
		\item Lists store elements in sequential locations in memory.
		\item Both stacks and queues allow us insert or delete an element at the front.
		\item We can use two queues to implement stack.
	\end{enumerate}
\end{Q}


\begin{Q}
	Which of the following is/are applications of queue and stack respectively?
	\begin{enumerate}[(A)]
		\item \textbf{Queue}: A resource shared by multiple users/processes; \textbf{Stack}: Handling function calls
		\item \textbf{Queue}: Loading Balancing; \textbf{Stack}: Reverse-Polish Notation
		\item \textbf{Queue}: Handling of interrupts in real-time systems; \textbf{Stack}: Compilers/Word Processors
		\item \textbf{Queue}: IO Buffers; \textbf{Stack}: Arithmetic expression evaluation
	\end{enumerate}
\end{Q}



\begin{Q}
	Read the following code, what function does it realize? \\
	\begin{lstlisting}[language=C++]
	  void Q4(Queue &Q) 
	  {
	    Stack S;
		int d;
		InitStack(S);
		while(!QueueEmpty(Q))
		{
		  DeQueue(Q,d)
		  Push(S,d);
		}
		while(!StackEmpty(S))
		{
		  Pop(S,d);
		  EnQueue(Q,d);
		}
	  }
	\end{lstlisting}
	\begin{enumerate}[(A)]
		\item Use stack to reverse the queue.
		\item Use queue to reverse the stack.
		\item Use stack to implement the queue.
		\item Use queue to implement the stack.
	\end{enumerate}
\end{Q}



\begin{Q}
	Which of the following comparison is correct?  
	\begin{enumerate}[(A)]
		\item $n^2 + n^3 = O(n^4)$
		\item  $\log_2 n = \Theta(\log n)$
		\item  $\log^2 n = \Omega(\log\log n)$
		\item $n! = \omega(n^n)$
	\end{enumerate}
\end{Q}


\section{(10') Stack and Queue}
\begin{Q}
\textbf{(2')}The following post-fix expression (Reverse-Polish Notation) with single digit operands is evaluated usng a stack:\\
$$8\ 2\ 3\ \hat\ /\ 2\ 3\ *\ +\ 5\ 1\ *\ - $$

Note that \^\ is the exponentiation operator. Please write down the corresponding in-fix notation A and the final result:\\
\vspace{0.5in}
\end{Q}


\begin{Q}
\textbf{(4')}\textbf{Describe} how to implement a queue using a singly-linked list. You can use pseudocode or natural language to describe all the operations, especially the key operations.
\end{Q}
\vspace{2.5in}
\begin{Q}
\textbf{(1')}If we use an array with size $N$ to implement a normal queue, it gets full when the index \textbf{Back} pointing to the index = \rule[-10pt]{2cm}{0.05em}
\vspace{0.1in}
\end{Q}

\begin{Q}
\textbf{(1')}By implementing the following  operations on stack, the value of x is \rule[-10pt]{2cm}{0.05em}\\
InitStack(st); Push(st,a); Push(st,b); Pop(st,x); Top(st,x);
\vspace{0.1in}
\end{Q}

\begin{Q}
\textbf{(2')}What dose "stack overflow" and "stack underflow" mean? (give a short explanation)
\vspace{1.5in}
\end{Q}



\section{(8') Complexity Analysis}
\begin{Q}
\textbf{(3')}Given a fraction of a code as the following, write down the time complexity for each \textbf{for} loop. 

\begin{lstlisting}[language=C++]
	for ( i=1; i<n; i*=2) {                    _______________________ 
	  for( j=n; j>0; j/=2) {                   _______________________ 
		  for ( k=j; k<n; k+=2) {          _______________________ 
			sum += (i + j*k)
		  }
	  }
	}
\end{lstlisting}
\end{Q}


\begin{Q}
\textbf{(5')}	Calculate the average processing time $T(n)$ of the following recursive algorithm. Suppose that it takes one unit time for \textbf{random( int n )} to return a random integer which is uniformly distributed in the range [0,n]. Also note that $T(0)=0$. \\
	\textit{Hints}:The equation $\frac{1}{1*2}+\frac{1}{2*3}+...+\frac{1}{n*(n+1)}=\frac{n}{n+1}$ might be needed.
	
	\begin{lstlisting}[language=C++]
	int hw( int n) {
	  if ( n <= 0 ) return 0;
	    else {
		  int i = random( n-1 );
		  return hw( i ) + hw( n-1-i );
		}
	}
	\end{lstlisting}
\end{Q}






\end{document}
